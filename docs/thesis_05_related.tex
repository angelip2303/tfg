\epigraph{\textit{If I have seen further, it is by standing on the shoulders of giants.}}{-- \textup{Isaac Newton}}

Some work has already been done in the field of Knowledge graph validation. In this section we are exploring what other projects have achieved and their limitations.

\section{Bid data processing and graphs}

Having to process enormous graphs made Google propose Pregel~\cite{10.1145/1807167.1807184} a model for large-scale graph computing, back in 2010. Following the idea of \textit{think like a vertex}, other systems were introduced: GraphLab~\cite{10.14778/2212351.2212354}, PowerGraph~\cite{180251} or GraphX~\cite{186216}. Being the latter a framework that enables the implementation of parallel computing algorithms.

\section{Knowledge graphs}

This article is closely related to Labra's paper~\cite{https://doi.org/10.48550/arxiv.2110.11709} on utilizing Shape Expressions to generate knowledge graph subsets, where he described the approach used in this document as one of the potential implementations. MARS (Multi-Attributed Relational Structures)~\cite{ijcai2017p165}, which are a generalized concept of property graphs, are the source of inspiration for our description of Wikibase graphs. They also define MAPL (Multi-Attributed Predicate Logic) as a formalism of logic that may be applied to ontological reasoning in that work.

\section{Knowledge graph descriptions}

\section{Knowledge graph subsets}