\epigraph{\textit{The programmers of tomorrow are the wizards of the future. You're going to look like you have magic powers compared to everybody else.}}{-- \textup{Gabe Newell}}

\section{Achievements}

From June the $25^{th}$ to July the $1^{st}$, the \textit{DBCLS BioHackathon 2023} took place, this is an event focused on the standardization and interoperability of life sciences and biomedical databases, namely, Knowledge Graphs. Following this, during the event, the tool presented in this project was used to create some subsets of the \textit{Uniprot} database. Which was later published in the \textit{Zenodo} repository, and can be found in the following link: \url{https://zenodo.org/record/8086938}.

\section{Future work}

The tool presented in this project is still in its early stages, and many improvements can be made to it. Some of the most important ones are:

\begin{enumerate}
    \itemsep0.5em
    \item \textbf{Improve the performance of the tool}: The tool is still slow, and it can be improved by using a more efficient programming language, or by using a more efficient algorithm.
    \item \textbf{Add more features}: The tool can be improved by adding more features, such as the ability to create subsets of the database based on the taxonomy of the proteins.
\end{enumerate}

Apart from the improvements that I have just mentioned, several other use cases can be implemented using the tool presented in this project. Some of the most important ones are:

\section{Personal opinion}