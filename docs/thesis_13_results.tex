\epigraph{\textit{80\% of a piece of software can be written in 20\% of the total allocated time. Conversely, the hardest 20\% of the code takes 80\% of the time.}}{-- \textup{Roger S. Pressman}}

\section{\texttt{wd2duckdb}}

As we have seen in the previous chapter, the \texttt{wd2duckdb} tool is analyzed thoroughly for us to understand how it performs and what can be done to improve it. In this section, we will present the execution times of the tool provided several configurations. What we want to see is how the tool performs with different optimizations enabled and how it compares to the original version of the tool. As such, we will prove that the optimizations we have implemented are indeed useful and that they improve the performance of the tool. Recall that the optimizations that we have implemented were described in section \ref{section:optimizations}.

\begin{figure}[p]
    \begin{subfigure}{0.49\textwidth}
        \centering
        \includestandalone[width=\textwidth]{diagrams/13-1_wd2duckdbOPT}
        \caption{Having all the optimizations enabled}
    \end{subfigure}%
    \hfill
    \begin{subfigure}{0.49\textwidth}
        \centering
        \includestandalone[width=\textwidth]{diagrams/13-2_wd2duckdbDEV}
        \caption{Having no optimization enabled}
    \end{subfigure}%
    \vspace*{1em}
    \begin{subfigure}{\textwidth}
        \centering
        \includestandalone[width=\textwidth]{diagrams/13-3_wd2duckdbBAR}
        \caption{Comparison between the two options: with and without optimizations}
    \end{subfigure}
    \caption{Time to create the database with \texttt{wd2duckdb}}
\end{figure}


\section{\texttt{pschema-rs}}
