\epigraph{\textit{It doesn't matter how beautiful your theory is, it doesn't matter how smart you are. If it doesn't agree with experiment, it's wrong.}}{-- \textup{Richard P. Feynman }}

\section{Knowledge graphs}

A knowledge graph uses a graph-structured data model to represent knowledge of some real-world domain. Where each node represents an entity -- a \textit{thing} from the actual world -- and the edges are the relationships between them. This type of graph is often assembled from a wide variety of sources, and as a result, can be highly diverse in terms of structure and granularity~\cite{DBLP:journals/corr/abs-2003-02320}. To address this issue, representations of \textit{schema}, \textit{identity} and \textit{context} are needed. While the former defines the high-level structure of the graph, \textit{identity} relates nodes that conform the same real-world entity; finally, \textit{context} provides the environment for a specific knowledge to be understood. These graphs can be modelled using different technologies; however, we will focus on the most representative example of attributed ones: \textit{Wikibase graphs}.

\subsection{Wikibase graphs}

Wikidata started back in 2012 as a support to Wikipedia. Rapidly becoming one of the biggest human knowledge bases, with remarkable organizations donating their data to it; as an example, Google migrated \textit{Freebase} -- its previous knowledge graph -- to Wikidata in 2017~\cite{10.1145/2872427.2874809}. As of October 9, 2022, Wikidata currently contains roughly 100 million items\footnote{\url{https://www.wikidata.org/wiki/Wikidata:Statistics}}. Internally, Wikidata's content is stored in a SQL database (MariaDB); however, this system is not ideal for querying or data analysis. With that in mind, and pursuing the integration of Wikibase within the semantic web ecosystem, the Wikimedia Foundation adopted BlazeGraph: an open-source triple store and graph database\footnote{\url{https://en.wikipedia.org/wiki/Blazegraph}}. This way, two data models coexist in Wikibase: a document-centric model based on MediaWiki, and an RDF-based one, that can be used to perform SPARQL queries through the Query Service~\cite{https://doi.org/10.48550/arxiv.2110.11709}.

\begin{figure}[h]
    \centering
    \includestandalone{figures/fig_02_architectureWikibase}
    \caption[Simplified architecture of Wikibase]{Simplified architecture of Wikibase~\cite{https://doi.org/10.48550/arxiv.2110.11709}}
    \label{fig:architecture:wikibase}
\end{figure}

Informally speaking, the Wikibase data model is composed by \textit{entities} and \textit{statements} (about those entities). This way, an entity can be either an \textit{item} or a \textit{property}. Items are used to represent all the \textit{things} from the human knowledge. Usually denoted by a \texttt{Q} followed by an identifier; for example, \href{https://www.wikidata.org/wiki/Q7251}{Q7251} represents Alan Turing in Wikidata. On the other hand, properties model a relationship between an item and a value, and are represented by a \texttt{P} followed by its identifier; as an example, \href{https://www.wikidata.org/wiki/Property:P31}{P31} is the property \textit{instance of} in Wikidata.

For values to be associated to properties, they must belong to some specific data type\footnote{\url{https://www.wikidata.org/wiki/Help:Data_type}}. Supported data types include: URLs, time, quantities, mathematical expressions; to name a few.

Lastly, an statement is a piece of data about an item\footnote{\url{https://www.wikidata.org/wiki/Help:Statements}}, and is recorded in the page of the item itself. Putting it all together, the way Wikibase manages information is as follows: through an statement, we add information to a certain item using some property and its value. An example to that is:

\begin{table}[h]
    \centering
    \begin{tabular}{|c|l|l|l|}
        \hline
        \rowcolor[HTML]{EFEFEF}
        \textbf{Language}     & \multicolumn{1}{c|}{\cellcolor[HTML]{EFEFEF}\textbf{Item}}                              & \multicolumn{1}{c|}{\cellcolor[HTML]{EFEFEF}\textbf{Property}}                            & \multicolumn{1}{c|}{\cellcolor[HTML]{EFEFEF}\textbf{Value}}                             \\ \hline
        \textit{English}      & \multicolumn{1}{l|}{\href{https://www.wikidata.org/wiki/Q11678}{The Hunger Games}}      & \multicolumn{1}{l|}{\href{https://www.wikidata.org/wiki/Property:P136}{genre}}            & \multicolumn{1}{l|}{\href{https://www.wikidata.org/wiki/Q15062348}{dystopian fiction}}  \\ \hline
        \textit{Spanish}      & \multicolumn{1}{l|}{\href{https://www.wikidata.org/wiki/Q11678}{Los juegos del hambre}} & \multicolumn{1}{l|}{\href{https://www.wikidata.org/wiki/Property:P136}{género}}           & \multicolumn{1}{l|}{\href{https://www.wikidata.org/wiki/Q15062348}{ficción distópica}}  \\ \hline
        \textit{French}       & \multicolumn{1}{l|}{ \href{https://www.wikidata.org/wiki/Q11678}{Hunger Games}}         & \multicolumn{1}{l|}{\href{https://www.wikidata.org/wiki/Property:P136}{genre artistique}} & \multicolumn{1}{l|}{\href{https://www.wikidata.org/wiki/Q15062348}{fiction dystopique}} \\ \hline
        \multicolumn{1}{l|}{} & \multicolumn{1}{c|}{\href{https://www.wikidata.org/wiki/Q11678}{Q11678}}                & \multicolumn{1}{c|}{\href{https://www.wikidata.org/wiki/Property:P136}{P136}}             & \multicolumn{1}{c|}{\href{https://www.wikidata.org/wiki/Q15062348}{Q15062348} }         \\ \cline{2-4}
    \end{tabular}
    \caption[Including information about the genre of \textit{The Hunger Games}]{\label{tab:wikibase:graph}Including information about the genre of \textit{The Hunger Games}\footnote{\url{https://www.wikidata.org/wiki/Q11678}}}
\end{table}

This way, what we are achieving is a database where we link concepts in a language-agnostic approach. Notice that in the previous example there is only one \textit{statement} which relates the \textit{item} \texttt{Q11678} with the \textit{value} \texttt{Q15062348} through the \textit{property} \texttt{P136}. Instead of storing attributes that are language-dependent; following this example, The \textit{statement} $s = \{ (\href{https://www.wikidata.org/wiki/Q11678}{Q11678}, \href{https://www.wikidata.org/wiki/Property:P136}{P136}, \href{https://www.wikidata.org/wiki/Q15062348}{Q15062348}, \{\}) \}$ is equivalent to: \textit{the genre of The Hunger games is dystopian fiction}. What we are doing is linking concepts.

\begin{definition}[Wikibase graph~\cite{https://doi.org/10.48550/arxiv.2110.11709}]
    Given a mutually disjoint set of items \ItemSet{}, a set of properties \PropSet{} and a set of data values \DataValueSet{}, a \emph{Wikibase graph} is a tuple $\langle\ItemSet,\PropSet,\DataValueSet,\StmtSet\rangle$ such that $\StmtSet\subseteq\EntitySet\times\PropSet\times\ValueSet\times\FinSet{\PropSet\times\ValueSet}$ where $\EntitySet=\ItemSet\cup\PropSet$ is the set of entities which can be subjects of a statement and $\ValueSet=\EntitySet\cup\DataValueSet$ is the set of possible values of a property.
\end{definition}

For better understanding this formal definition, let me explain it step-by-step. First, let us clarify what the described symbols are: $\mathcal{Q}$ is the set containing the different entities (\textit{things}), $\mathcal{P}$ is the set with the different possible relationships in the graph, $\mathcal{D}$ is the set containing data values and $\rho$ is the set where each element represents a relation between two entities and a certain data value. Having that said, we can now see an example for further clarifications:

\begin{example}
    We are willing to qualify that Alan Turing (23 June 1912 -- 7 June 1954) was employed by the government of the United Kingdom in the course of the WWII. During that time he invented a computer for deciphering Enigma-machine-encrypted secret messages\footnote{\url{https://en.wikipedia.org/wiki/Bombe}}. Additional information about relevant places where he lived is also annotated.
\end{example}

\begin{table}[h]
    % Longatable File '01_tab_wikibaseGraph.tex'
\documentclass{standalone}
\usepackage{longtable}
\newcommand{\myfont}[1]{\ensuremath{\mathcal{#1}}}

\newcommand{\triple}[3]{\ensuremath{\langle #1,#2,#3 \rangle}}
\newcommand{\quadruple}[4]{\ensuremath{\langle #1,#2,#3,#4\rangle}}

\newcommand{\ItemSet}{\myfont{Q}}
\newcommand{\PropSet}{\myfont{P}}
\newcommand{\EntitySet}{\myfont{E}}
\newcommand{\ValueSet}{\myfont{V}}
\newcommand{\DataValueSet}{\myfont{D}}
\newcommand{\StmtSet}{\rho}
\newcommand{\FinSet}[1]{\ensuremath{FinSet(#1)}}

\newcommand{\hrefc}[3][blue]{\href{#2}{\color{#1}{#3}}}%

\newcommand{\elemento}[2]{\ensuremath{\hrefc[violet]{http://www.wikidata.org/entity/#2}{#1}}}
\newcommand{\propiedad}[2]{\ensuremath{\hrefc[blue]{http://www.wikidata.org/entity/#2}{#1}}}
\newcommand{\alanTuring}{\elemento{alanTuring}{Q7251}}
\newcommand{\wilmslow}{\elemento{wilmslow}{Q2011497}}
\newcommand{\town}{\elemento{town}{Q3957}}
\newcommand{\government}{\elemento{government}{Q220798}}
\newcommand{\warringtonLodge}{\elemento{warringtonLodge}{Q20895942}}
\newcommand{\bombe}{\elemento{bombe}{Q480476}}
\newcommand{\unitedKingdom}{\elemento{unitedKingdom}{Q145}}
\newcommand{\computer}{\elemento{computer}{Q11742076}}
\newcommand{\dateOfBirth}{\propiedad{dateOfBirth}{P569}}
\newcommand{\placeOfBirth}{\propiedad{placeOfBirth}{P19}}
\newcommand{\country}{\propiedad{country}{P27}}
\newcommand{\employer}{\propiedad{employer}{P108}}
\newcommand{\discoverer}{\propiedad{discoverer}{P61}}
\newcommand{\dateOfDeath}{\propiedad{dateOfDeath}{P570}}
\newcommand{\placeOfDeath}{\propiedad{placeOfDeath}{P20}}
\newcommand{\timeStart}{\propiedad{timeStart}{P580}}
\newcommand{\timeEnd}{\propiedad{timeEnd}{P582}}
\newcommand{\manufacturer}{\propiedad{manufacturer}{P176}}
\newcommand{\Human}{\elemento{Human}{Q5}}
\newcommand{\instanceOf}{\propiedad{instanceOf}{P31}}

\setlength{\tabcolsep}{1mm}
\renewcommand{\arraystretch}{1.5}
\begin{document}
\begin{longtable}{ccl}
    \ItemSet      & = \{ & \alanTuring, \wilmslow, \government, \unitedKingdom, \warringtonLodge,                           \\
                  &      & \bombe, \town, \computer \hspace{1mm} \}                                                         \\
    \PropSet      & = \{ & \dateOfBirth, \placeOfBirth, \employer, \dateOfDeath, \placeOfDeath,                             \\
                  &      & \timeStart, \country, \timeEnd, \manufacturer, \instanceOf \hspace{1mm} \}                       \\
    \DataValueSet & = \{ & {\small 23 June 1912}, {\small 7 June 1954}, {\small 1938}, {\small 1945}\hspace{1mm} \}         \\
    $\StmtSet$    & = \{ & (\alanTuring, \instanceOf, \Human, \{\}),                                                        \\
                  &      & (\alanTuring, \dateOfBirth, {\small 23 June 1912}, \{\}),                                        \\
                  &      & (\alanTuring, \placeOfBirth, \warringtonLodge, \{\}),                                            \\
                  &      & (\alanTuring, \dateOfDeath, {\small 7 June 1954}, \{\}),                                         \\
                  &      & (\alanTuring, \placeOfDeath, \wilmslow, \{\}),                                                   \\
                  &      & (\alanTuring, \employer, \government, \{ \timeStart: {\small 1938}, \timeEnd: {\small 1945} \}), \\
                  &      & (\bombe, \discoverer, \alanTuring, \{\}),                                                        \\
                  &      & (\bombe, \manufacturer, \government, \{\}),                                                      \\
                  &      & (\bombe, \instanceOf, \computer, \{\}),                                                          \\
                  &      & (\bombe, \country, \unitedKingdom, \{\})                                                         \\
                  &      & (\wilmslow, \country, \unitedKingdom, \{\})                                                      \\
                  &      & (\wilmslow, \instanceOf, \town, \{\})                                                            \\
                  &      & (\warringtonLodge, \country, \unitedKingdom, \{\})\hspace{1mm} \}                                \\
\end{longtable}
\end{document}
    \caption{Example of a Wikibase graph according to its definition}
    \label{tab:wikibase:graph}
\end{table}

The figure \ref{fig:wikibase:graph} shows a visualization of the previously described \ref{tab:wikibase:graph} Wikibase graph.

\begin{figure}[h]
    \centering
    \includestandalone{figures/fig_03_wikibaseGraph}
    \caption[Visualization of an example of a Wikibase knowledge graph]{Visualization of the previous example of a Wikibase knowledge graph\footnote{\url{https://www.wikidata.org/wiki/Q7251}}}
    \label{fig:wikibase:graph}
\end{figure}

\subsubsection{Wikibase data serialization}

The Wikibase data model supports two export formats: JSON and RDF. Being the former the one employed by the JSON dumps and the latter follows semantic web and linked data principles.

\paragraph{Wikibase JSON serialization}

is represented as:

\begin{lstlisting}[style=JSON]
[
    { "type": "item", "id": "Q42", "claims": { "P31": [...
    { "type": "item", "id": "Q80", "claims": { "P108": [...
    { "type": "property", "id": "P108", "claims": { ... 
    ... 
] 
\end{lstlisting}

\paragraph{Wikibase RDF serialization}

is represented as\footnote{\url{https://www.wikidata.org/wiki/Special:EntityData/Q7251.ttl}}:

\begin{lstlisting}[style=Turtle]
wd:Q7251 p:P108 s:Q7251-5c5c1b1c-40a4-b9c0-40a6-dfaccd56f273 .

s:Q7251-5c5c1b1c-40a4-b9c0-40a6-dfaccd56f273 a wikibase:Statement,
wikibase:BestRank ;
wikibase:rank wikibase:NormalRank ;
ps:P108 wd:Q220798 ;
pq:P580 "1938-01-01T00:00:00Z"^^xsd:dateTime ;
pqv:P580 v:560845b053d80d9ba30cc757808172fd ;
pq:P582 "1945-01-01T00:00:00Z"^^xsd:dateTime ;
pqv:P582 v:db243e11a2e2c6bb034f59e082f66d29 ;
prov:wasDerivedFrom ref:7d58b689209415697990098179e8ace8b04630a1 .
\end{lstlisting}

\section{Data-flow algorithms}

The main focus of this document is implementing a Big data solution using the Pregel algorithm. However, for better understanding it, introducing the MapReduce model will be handful. Notice that both are meant to be executed in parallel in a distributed system.

\subsection{MapReduce}

Inspired by the map and reduce functions, a MapReduce~\cite{wiki:MapReduce} program is composed of a \textit{map procedure}, where we apply a simple operation to all the elements of a sequence, followed by a \textit{reduce} method, which transforms those elements into a single result. For us to process a graph, we would need to chain MapReduce invocations. Where, for each iteration, map and reduce functions are applied. The main drawback of this approach, is the functional nature of the MapReduce model. This means, expressing a graph algorithm as a chained MapReduce results in having to pass the entire state of the graph from one stage to another~\cite{10.1145/1807167.1807184} .

\subsection{Pregel model}

Pregel (\textit{Parallel, Graph and Google}) is a data flow paradigm and system created by Google to handle large-scale graphs. Even though the original system remains proprietary at Google, the computational model was adopted by many graph-processing systems: including Apache Spark. For better understanding Pregel, the idea is to \textit{think like a vertex}; this way, for computing the state of a given node, we only depend on the states of its neighboring\footnote{We will call neighboring vertices of a certain one to those nodes connected to it by an outgoing edge. This has a relevance for better understanding the architecture of this system ~\ref{section:architecture:pregel}.} ones. \textit{Thinking like a vertex} could be understood as the \textit{leitmotif} for dividing the problem into several sub-problems: instead of dealing with a huge graph\footnote{A graph to be processed with Pregel may potentially have millions of vertices with billions of edges. More on the size of the Wikibase graphs will be discussed later on.}, we just have to solve the problem for smaller graphs: a vertex and its neighboring ones.

In comparison to the MapReduce framework, where for each iteration the state of the whole graph must be passed, at each \textit{superstep} -- the way we refer to iterations in Pregel -- each vertex can: send a message to its neighbors, process the received messages (from the previous superstep), and update its state. Summing up, instead of sending the whole state of the graph, we just send messages back and forth. The best way for understanding this is through an example. Let me show you the resulting trace after applying Pregel for computing the maximum value in a graph (see figure \ref{fig:pregel}).

\begin{figure}[h]
    \centering
    \includestandalone{figures/fig_04_pregelTrace}
    \caption[Trace of the execution of Pregel for computing the maximum value]{Trace of the execution of Pregel for computing the maximum value~\cite{10.1145/1807167.1807184}}
    \label{fig:pregel}
\end{figure}

Notice that at the beginning of the execution, the state of all the nodes will be set to active. This initial stage is called \textit{superstep 0}. When a vertex is active, it sends a message to its neighbors, that will receive it in the next \textit{superstep}. In this case, the message we are propagating is the largest value that we have learned so far. As an example to that, the second to the left node starts with a value of 6 -- which is the maximum value of the graph -- and sends that number to its neighbors: 3 and 1. In the next \textit{superstep}, the vertices that have received a message have to compare both: the value they store and the received one. This comparison is what it's called the \textit{vprog} function, which will vary from one problem to another. Then, both nodes will update its state to 6. As they have updated their state, they will have to send the new value to its neighbors: beginning another iteration (\textit{superstep}). Notice that at the \textit{superstep 1}, the second to the left node is halted as it doesn't send any other message to its neighbors as it's value has not changed. When every node is halted -- inactive -- the execution of the algorithm finishes. In the figure \ref{fig:state} a simplified state machine is shown.

\begin{figure}[h]
    \centering
    \includestandalone{figures/fig_05_stateDiagram}
    \caption[Simplified state diagram of a vertex]{Simplified state diagram of a vertex~\cite{10.1145/1807167.1807184}}
    \label{fig:state}
\end{figure}

\subsubsection{Architecture of a Pregel system}
\label{section:architecture:pregel}

It's quite simple to describe the architecture of a Pregel system. As we have seen, one of the main goals of this algorithm is achieving a parallel execution. This way, a \textit{master} node will divide the graph into several partitions and assign one (or more) of them to each \textit{worker} node. For the \textit{master} to create the partitions it selects a bunch of vertices and all those vertices' outgoing edges; remember: \textit{think like a vertex}.

\begin{figure}[h]
    \centering
    \includestandalone{figures/architecture_pregel}
    \caption[Architecture of a Pregel system]{Architecture of a Pregel system~\cite{10.1145/3349265}}
    \label{fig:architecture:pregel}
\end{figure}
